\documentclass[11pt,a4paper,english]{article}
%\usepackage[ngerman]{babel} % \usepackage[shortcuts]{extdash} % Explicit Hyphenation: multi\-/disciplinary. Deny hypen: multi\=/disciplinary
\usepackage[english]{babel}
\usepackage{MA_Titlepage}

\usepackage{savefnmark} % Reusing a footnote with '\saveFN\fnname' and '\useFN\fnname'

\usepackage{hyperref}
% Allgemeines
% \usepackage[automark]{scrpage2} % Kopf- und Fußzeilen
\usepackage{amsmath, marvosym, amssymb} % Mathesachen % http://ctan.org/pkg/amsmath
\usepackage{blkarray} % Matrix % http://www.hss.caltech.edu/~kcb/TeX/kbordermatrix.sty
\usepackage{siunitx}%Schöne Darstellung großer Zahlen \num{}
\usepackage[T1]{fontenc} % Ligaturen, richtige Umlaute im PDF
\usepackage[utf8]{inputenc}% UTF8-Kodierung für Umlaute usw
\usepackage[normalem]{ulem} % for strike outs: \sout{Crossed out text}
% Schriften
\usepackage{mathpazo} % Palatino für Mathemodus
\usepackage{mathtools}% Fur {}pmatrix*}[r]
%\usepackage{mathpazo,tgpagella} % auch sehr schöne Schrift
% Schriften-Größen
% Sprache: Deutsch
% PDF
\usepackage{pdflscape} % einzelne Seiten drehen können
% Tabellen
\usepackage{multirow} % Tabellen-Zellen über mehrere Zeilen
\usepackage{multicol} % mehrere Spalten auf eine Seite
\usepackage{tabularx, booktabs} % für Tabellen mit vorgegeben Größen
\usepackage{longtable} % Tabellen über mehrere Seiten
\usepackage{array}
%  Bibliographie
\usepackage{bibgerm} % Umlaute in BibTeX
% Bilder
\usepackage{graphicx} % Bilder
% \usepackage{color} % Farben
\usepackage[usenames, dvipsnames]{color}
\graphicspath{{images/}} % Lege den Standardpfad mit Bilder fest
\DeclareGraphicsExtensions{.pdf,.png,.jpg} % bevorzuge pdf-Dateien vor den anderen
%\usepackage[justification=RaggedRight, singlelinecheck=false]{caption}
\usepackage[justification=centering]{caption}
\usepackage{subcaption}  % mehrere Abbildungen nebeneinander/übereinander
% Für Pseudocode-Darstellungen
\usepackage{algorithm}
\usepackage{algpseudocode} 
\makeatletter
\def\BState{\State\hskip-\ALG@thistlm}
\algdef{SE}[DOWHILE]{Do}{doWhile}{\algorithmicdo}[1]{\algorithmicwhile\ #1}
\makeatother
% Quellcode
% für Formatierung in Quelltexten, hier im Anhang
\usepackage{listings}
\usepackage{csquotes} % für Anführungszeichen/ Zitate
\usepackage{enumerate}
\usepackage{enumitem}
\usepackage{minted}
\usepackage{tikz} % Graph construction 
\setminted[python]{breaklines, frame=lines, framesep=2mm, fontsize=\footnotesize, numbersep=5pt, linenos}
%\usepackage[shortlabels]{enumerate}
\definecolor{grau}{gray}{0.25}
\lstset{
	extendedchars=true,
	basicstyle=\tiny\ttfamily,
	%basicstyle=\footnotesize\ttfamily,
	tabsize=2,
	keywordstyle=\textbf,
	%commentstyle=\color{grau},
	stringstyle=\textit,
	numbers=left,
	numberstyle=\tiny,
	% für schönen Zeilenumbruch
	breakautoindent  = true,
	breakindent      = 2em,
	breaklines       = true,
	postbreak        = ,
	prebreak         = \raisebox{-.8ex}[0ex][0ex]{\Righttorque},
}

\definecolor{javared}{rgb}{0.6,0,0} % for strings
\definecolor{javagreen}{rgb}{0.25,0.5,0.35} % for strings
\definecolor{javapurple}{rgb}{0.5,0,0.35} % for strings
\definecolor{javadocblue}{rgb}{0.25,0.35,0.75} % for strings
\lstset{
	language=Java,
	basicstyle=\fontsize{8}{7}\ttfamily,
	keywordstyle=\color{javapurple}\bfseries,
	stringstyle=\color{javared},
	commentstyle=\color{javagreen},
	morecomment=[s][\color{javadocblue}]{/**}{*/}
	numbers=left,
	numberstyle=\tiny\color{black},
	stepnumber=2,
	numbersep=10pt,
	tabsize=4,
	showspaces=false,
	showstringspaces=false
%	{Ö}{{\"O}},	{Ä}{{\"A}},	{Ü}{{\"U}},	{ß}{{\ss}},	{ö}{{\"o}},	{ä}{{\"a}},	{ü}{{\"u}},
}
%\lstset{
%	literate=%
%	{Ö}{{\"O}}1
%	{Ä}{{\"A}}1
%	{Ü}{{\"U}}1
%	{ß}{{\ss}}1
%	{ö}{{\"o}}1
%	{ä}{{\"a}}1
%	{ü}{{\"u}}1
%}

\usepackage{scrhack} % Vermeidung einer Warnung

%%% Numbersets
\newcommand{\eye}{\mathbb{1}} % Identitiy matrix
\newcommand{\one}{\textbf{1}} % One-vector (1 1 ... 1)
\newcommand{\IN}{\mathbb{N}} % Natural numbers
\newcommand{\IR}{\mathbb{R}} % Real numbers
\newcommand{\IZ}{\mathbb{Z}} % Integers
\newcommand{\IQ}{\mathbb{Q}} % Rational numbers
\newcommand{\ID}{\mathbb{D}} % Dyadic numbers
\newcommand{\IC}{\mathbb{C}} % Complex numbers
\newcommand{\IF}{\mathbb{F}} % (Vector) Field
\newcommand{\IW}{\mathbb{W}} % Wasserstein distance
\newcommand{\vA}{\mathcal{A}}
\newcommand{\vD}{\mathcal{D}}
\newcommand{\vB}{\mathcal{B}}
\newcommand{\vP}{\mathcal{P}}
\newcommand{\vJ}{\mathcal{J}}
\newcommand{\vU}{\mathcal{U}}
%%% Basic operators
\newcommand{\IP}{\mathbb{P}} % Probability operator
\newcommand{\IE}{\mathbb{E}} % Expectation operator
\newcommand{\vF}{\mathcal{F}} % Fourier transform
%%% Distributions
\newcommand{\vN}{\mathcal{N}} % Normal distribution
\newcommand{\Bin}{\mathop{\mathrm{Bin}}} % Binomial distribution
\newcommand{\Poi}{\mathop{\mathrm{Poi}}} % Poisson distribution
%%% Others
\newcommand{\Cov}{\mathop{\mathrm{Cov}}} % covariance
\newcommand{\co}{\mathop{\mathrm{co}}} % covariance
\newcommand{\Var}{\mathop{\mathrm{Var}}} % variance
\newcommand{\norm}[1]{\left\lVert#1\right\rVert} % norm
\newcommand{\var}[1]{{\ttfamily#1}} % variable
\newcommand{\tr}{\mathop{\operatorname{tr}}} % trace
\newcommand{\rk}{\mathop{\operatorname{rk}}} % rank
\newcommand{\diag}{\mathop{\operatorname{diag}}} % diagonal
\renewcommand{\vec}[1]{\bm{#1}} % vector
\newcommand{\mat}[1]{\bm{#1}} % matrix
\newcommand{\ten}[1]{\bm{\mathcal{#1}}}
\newcommand{\inv}[1]{#1^{-1}} % inverse
\newcommand{\trn}[1]{#1^\intercal} % transpose
\newcommand{\opt}[2]{#1 \trn{#2}} % outer product
\newcommand{\ipt}[2]{\trn{#1} #2} % inner product
\newcommand{\angles}[2]{\langle #1, #2 \rangle} % scalar product
\newcommand{\Angles}[2]{\bigl \langle #1, #2 \bigr \rangle} % big scalar product
\newcommand{\st}{\operatorname{s.\!t.}} % 'such that'
\DeclareMathOperator{\Ima}{\textit{Im}}
%%% Stacked symbols
\newcommand{\amin}[1]{\operatorname*{argmin}_{#1}}
\newcommand{\amax}[1]{\operatorname*{argmax}_{#1}}
\newcommand{\sign}{\mathop{\mathrm{sign}}}
\newcommand{\eqex}{\mathop{\stackrel{!}{=}}}
\newcommand{\geex}{\mathop{\stackrel{!}{\ge}}}
\newcommand{\leex}{\mathop{\stackrel{!}{\le}}}
\newcommand{\softmax}{\operatorname*{softmax}}
%%% Table colors
\newcommand\cellr{\cellcolor{red!20}}
\newcommand\cellg{\cellcolor{green!10}}
\newcommand\cellb{\cellcolor{blue!10}}
\newcommand\cello{\cellcolor{orange!10}}
%%% Sets
\newcommand{\set}[1]{\{{#1}\}}
\newcommand{\Set}[1]{\big\{{#1}\big\}}
\newcommand{\BigSet}[1]{\Big\{{#1}\Big\}}
%%% Multisets
\newcommand{\mset}[1]{\{\mskip-5mu\{{#1}\}\mskip-5mu\}}
\newcommand{\Mset}[1]{\big\{\mskip-5mu\big\{{#1}\big\}\mskip-5mu\big\}}
\newcommand{\BigMset}[1]{\Big\{\mskip-5mu\Big\{{#1}\Big\}\mskip-5mu\Big\}}
%%% Footnote
%\deffootnote{1.5em}{1em}{\makebox[1.5em][l]{\thefootnotemark}}


\newcommand{\qedwhite}{\hfill \ensuremath{\Box}} %QED Box

\newcommand{\todo}[1]{
      {\colorbox{BurntOrange}{ TODO: #1 }}
}
\newcommand{\info}[1]{
      {\colorbox{blue}{ (INFO: #1)}}
}
% Hinweis auf Programme in Datei
\newcommand{\datei}[1]{
      {\ttfamily{#1}}
}
% Hinweis auf Code in Datei
\newcommand{\code}[1]{
      {\ttfamily{#1}}
}
% bild mit defnierter Breite einfügen
%\newcommand{\bild}[4]{
%  \begin{figure}[!hbt]
%    \centering
%      \vspace{1ex}
%      \includegraphics[width=#2]{images/#1}
%      \caption[#4]{\label{img.#1} #3}
%    \vspace{1ex}
%  \end{figure}
%}
%% bild mit eigener Breite
%\newcommand{\bilda}[3]{
%  \begin{figure}[!hbt]
%    \centering
%      \vspace{1ex}
%      \includegraphics{images/#1}
%      \caption[#3]{\label{img.#1} #2}
%      \vspace{1ex}
%  \end{figure}
%}
%
%% Bild todo
%\newcommand{\bildt}[2]{
%  \begin{figure}[!hbt]
%    \begin{center}
%      \vspace{2ex}
%	      \includegraphics[width=6cm]{images/todobild}
%      %\caption{\label{#1} \color{red}{ TODO: #2}}
%      \caption{\label{#1} \todotext{#2}}
%      %{\caption{\label{#1} {\todo{#2}}}}
%      \vspace{2ex}
%    \end{center}
%  \end{figure}
%}
% Snippet to allow latex to interpret commas in math modes as possible breaking points.
% This reduces problems where math mode stops LaTex from suitable line breaks.
%\makeatletter 
%\def\old@comma{,}
%\catcode`\,=13
%\def,{%
%	\ifmmode%
%	\old@comma\discretionary{}{}{}%
%	\else%
%	\old@comma%
%	\fi%
%}
%\makeatother